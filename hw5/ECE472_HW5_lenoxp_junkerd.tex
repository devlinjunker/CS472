\documentclass[letterpaper,10pt,titlepage]{article}

\usepackage{graphicx}
\usepackage{amssymb}
\usepackage{amsmath}
\usepackage{amsthm}

\usepackage{alltt}
\usepackage{float}
\usepackage{color}
\usepackage{url}

\usepackage{balance}
\usepackage[TABBOTCAP, tight]{subfigure}
\usepackage{enumitem}
\usepackage{pstricks, pst-node}

\usepackage{geometry}
\geometry{textheight=9in, textwidth=6.5in}

%random comment

\newcommand{\cred}[1]{{\color{red}#1}}
\newcommand{\cblue}[1]{{\color{blue}#1}}

\usepackage{hyperref}
\usepackage{geometry}

\def\name{Philip Lenox, Devlin Junker}

%% The following metadata will show up in the PDF properties
\hypersetup{
  colorlinks = true,
  urlcolor = black,
  pdfauthor = {\name},
  pdfkeywords = {cs472 ``computer architecture'' clements ``chapter 1''},
  pdftitle = {CS 472: Homework 5},
  pdfsubject = {CS 472: Homework 5},
  pdfpagemode = UseNone
}

\begin{document}
\hfill \name

\hfill \today

\hfill CS 472 HW 5

\begin{enumerate}

\item[$(6.5)$]

The time taken by Machines A, B and C to exectue a given task is:\\
A 16m, 9s : 9s / 16m = 0.5625\\
B 14m, 12s : 12s / 14m = 0.8571\\
C 12m, 47s : 47s / 12m = 3.916\\
relative to A:\\
A = 1 A\\
B = A/B = 0.6563 A\\
C = C/A = 0.1426 A \\

\item[$(6.6)$] 

Clock rate is a poor metric for performance because clock rates can be clocked arbitrarily quickly, but if it takes more than 1 clock cycle to execute an instruction, the performance can go down while increasing the clock rate. Clock speed, if correctly chosen for the processor, can indicate the speed with which one instruction will complete, so prior to multi-core and super scalar processors, it was an ok metric for performance of a processor. 

\item[$(6.12)$] Would lowering the clock frequency by 15\% be a good idea?

To answer this question, we will evaluate the performance as is, and with the 15\% reduction \\
As is: $1(.45) + 3(.2) + 2(.1) + 2(.25) = 1.75$ \\
15\% reduction: $1(1(.45) + 2(.2) + 2(.1) + 2(.25))/(.85) = 1.823$\\
No, not worth it

\item[$(6.13)$] Determine cyles per conditional branch for 20\% improvement

current performance: $1(.65) + 5(.1) + 2(.05) + 8(.2) = 2.85 $\\
Target performance: current performance * 0.8 = 2.28 \\
2.28 - [1(.65) + 5(.1) + 2(.05)] = x(.2) \\
x = 5.15 \\
target cycles per conditional branch = 5\\

\item[$(6.17)$]

$ S = \frac{1}{f_{s} + \frac{1-f_{s}}{P}} $ \\ 
\begin{enumerate}
\item 10 processors , $f_{s} = 0.1: P = 10 : S =  \frac{1}{0.1 + \frac{1-0.1}{10}} = 5.26$
\item 100 processors , $f_{s} = 0.1: P = 100 : S =  \frac{1}{0.1 + \frac{1-0.1}{100}} = 9.17$
\item 5 processors , $f_{s} = 0.4: P = 5 : S =  \frac{1}{0.4 + \frac{1-0.4}{5}} = 1.92$
\item 100 processors , $f_{s} = 0.01: P = 100 : S =  \frac{1}{0.01 + \frac{1-0.01}{100}} = 50.25$
\end{enumerate}

\item[$(6.18)$]

The optimal number of cores to purchase is 32. Purchasing additional processors increase's the ratio by less than 5\% of the original speed. This would constitute a 306\% increase in the cost.

\begin{tabular}{ll}
Number of Processors 	&	Speedup Ratio 	\\
\hline
	1					&	1.00			\\
	2					&	1.82			\\
	3					&	2.50			\\
	4					&	3.08			\\
	5					&	3.57			\\
	10					&	5.26			\\
	20					&	6.90			\\
	30					&	7.69			\\
	31					&	7.75			\\
	32					&	7.80			\\
\end{tabular}

\item[$(6.22)$]

$\frac{1}{(1 - x) + \frac{x}{3.5}} = 1.5$
  
$x = 47\%$ of the code must be using the string processing instructions.

\item[$(7.12)$]

The memory, registers and ALU micro-operations could be performed for other or the next instructions while the conditional branch is being performed. 

\item[$(7.15)$]

Another adder would need to be added after the ALU\_MPLX before the ALU. The adder would have to be conected to the IR to recieve the shift operand.

\end{enumerate}
\end{document}


