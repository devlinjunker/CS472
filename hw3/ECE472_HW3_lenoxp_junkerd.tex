\documentclass[letterpaper,10pt,titlepage]{article}

\usepackage{graphicx}                                        
\usepackage{amssymb}                                         
\usepackage{amsmath}                                         
\usepackage{amsthm}                                          

\usepackage{alltt}                                           
\usepackage{float}
\usepackage{color}
\usepackage{url}

\usepackage{balance}
\usepackage[TABBOTCAP, tight]{subfigure}
\usepackage{enumitem}
\usepackage{pstricks, pst-node}

\usepackage{geometry}
\geometry{textheight=9in, textwidth=6.5in}

%random comment

\newcommand{\cred}[1]{{\color{red}#1}}
\newcommand{\cblue}[1]{{\color{blue}#1}}

\usepackage{hyperref}
\usepackage{geometry}

\def\name{Philip Lenox, Devlin Junker}

%% The following metadata will show up in the PDF properties
\hypersetup{
  colorlinks = true,
  urlcolor = black,
  pdfauthor = {\name},
  pdfkeywords = {cs472 ``computer architecture'' clements ``chapter 1''},
  pdftitle = {CS 472: Homework 2},
  pdfsubject = {CS 472: Homework 2},
  pdfpagemode = UseNone
}

\begin{document}
\hfill \name

\hfill \today

\hfill CS 472 HW 2

\begin{enumerate}



\item[$(3.1)$]Why is the program counter a pointer and not a counter?
	 
	 Counters are realtive to a certain point, which would have to be specified in another point. The pointer is similar to a counter, only with the starting point located at the start or end of memory depending on the system. A pointer is also used because of branch / jump statements that cause the pointer to move in a non-sequential manner.
	
\item[$(3.2)$]Expalin the function of the following registers

\begin{enumerate}
\item PC: program counter, keeps track of the location of the next instruction to be executed
\item MAR: Memory Address Register: stores the address of the data to be retrieved or stored
\item MBR: Memory Buffer Register: stores the data to be loaded or stored
\item IR: Instruction Register: stores the current fetched instruction 
\end{enumerate}

\item[$(3.3)$]

\begin{enumerate}
\item c =0 z =0 v=0 n=0
\item c =1 z =1 v=0 n=0
\item c =0 z =0 v=0 n=0
\item c =1 z =0 v=0 n=0
\item c =0 z =0 v=0 n=1
\item c =1 z =0 v=1 n=0
\end{enumerate}

\item[$(3.10)$]Why does ARM implement a reverse subtract?

Because literals can only be stored operand 2, so this allows the programmer to subtract from a literal.
\item[$(3.17)$]What are the advantages and disadvantages of of storing the 12 bit literal as an 8 bit literal and a 4 bit shift?

This allows for a greater range of values the literal, but gives less precision, especially when moving further from zero.
\item[$(3.18)$] Whrite one or more ARM instructions that will clear bits 20 to 25 inclusive in register 0.

$AND R0, R0, \#0xfe0fffff$
\item[$(3.19)$]Swap without an extra register.

$EOR R0, R0, R1
EOR R1, R0, R1
EOR R0, R0, R1$
\item[$(3.25)$]

\begin{enumerate}
\item xxxx xxxx xxxx xxxx xxxx xxxx xxxx xxxx
\item xxxx xxxx xxxx xxxx xxxx xxxx xxxx xxxx
\item xxxx xxxx xxxx xxxx xxxx xxxx xxxx xxxx
\item xxxx xxxx xxxx xxxx xxxx xxxx xxxx xxxx
\end{enumerate}
\item[$(3.39)$]
$START:
LDRB r2, [r0],\#1
STRB r2, [r1],\#1
TEQ R2, \#0
BRNE START
$
\item[$(3.51)$]Why is the program counter a pointer and not a counter?	 
\end{enumerate}
$START:
MOV r0, \#1
TEQ r1, r2
BREQ END
LDRB r3, [r1], \#1
LDRB r4, [r2], \#-1
TEQ r3, r4
BRNE NO
B START$
$
NO:
mov r0, \#0
END:
$
\end{document}
